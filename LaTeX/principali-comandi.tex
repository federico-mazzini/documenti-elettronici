\documentclass[14pt]{extreport}

% Lingua e codifica
\usepackage[italian]{babel}
\usepackage[letterpaper,top=2cm,bottom=2cm,left=3cm,right=3cm]{geometry}
\usepackage{graphicx} 
\usepackage[hidelinks]{hyperref}
\usepackage{setspace}
\usepackage{titlesec}
\usepackage{fancyhdr} % intestazioni e piè di pagina

% Capitoli con numero inline e iniziano sempre in pagina nuova
\titleformat{\chapter}[hang]
  {\normalfont\huge\bfseries}
  {\thechapter.}
  {1em}
  {} 
\titlespacing*{\chapter}{0pt}{0pt}{20pt} % spazio prima/dopo titolo

% Interlinea
\onehalfspacing

% Stile intestazione/piede di pagina
\pagestyle{fancy} 
\fancyhf{}
\fancyfoot[C]{\thepage}

\begin{document}

% --- Pagina iniziale ---
\begin{titlepage}
    \centering
  
    % Nome della scuola in grande
    {\LARGE \textbf{IIS ALBERGHETTI} \par}
    {\LARGE \textbf{Liceo Scientifico Scienze Applicate} \par}
    \vspace{2cm}

    % Logo scuola
    \vspace{3cm}

    % Titolo relazione
    {\LARGE \textbf{Template di documento LaTeX e principali comandi} \par}
    {\Large Informatica \par}
    {\large \today \par}
    
    % Nome studente a fondo pagina
    \vfill
    {\Large Nome studente\par}
    {\large 2CLS - 2ELS \par}

\end{titlepage}

\newpage

% Indice
\vspace*{5cm}
\tableofcontents
\newpage

% --- Capitoli ---
\chapter{Introduzione a capitoli e sezioni}
Testo introduttivo della relazione.

\section{Prima sezione del capitolo 1}
Una section è una sottoparte del capitolo (chapter).


\subsection{Sottosezione}
Questa è una sottosezione, sottoparte della sezione, quindi un paragrafo più annidato.

\subsubsection{Sezione ancora più annidata}
In questa sezione non c'è numerazione. Viene utilizzata se abbiamo necessità di una sottosezione aggiuntiva

\paragraph{Paragrafo} 
Questo è un paragraph

\section{Seconda sezione del capitolo 1}
Breve biografia dei principali pionieri dell'informatica.

\chapter{Esempi pratici}

\section{Elenchi numerati e puntati}
Elenchi numerati: 

\begin{enumerate}
\item Primo elemento
\item Secondo elemento
\end{enumerate}
Elenchi puntati:

\begin{itemize}
\item Punto 1
\item Punto 2
\end{itemize}

\subsection{Elenchi annidati}
\begin{itemize}
    \item Primo livello
    \item Ancora primo livello
    \begin{itemize}
        \item Secondo livello
        \item Ancora secondo livello
        \begin{itemize}
            \item Terzo livello
        \end{itemize}
    \end{itemize}
    \item Di nuovo primo livello
\end{itemize}

\newpage
\section{Inserire figure}
Per inserire immagini nel documento, caricare il file e usare il comando \verb|\includegraphics|.

\begin{figure}[h!] 
\includegraphics[width=0.25\linewidth]{immagini/immagine.jpg}
\caption{Esempio di figura}
\end{figure}


\begin{figure}[h!] 
\centering 
\includegraphics[width=0.25\linewidth]{immagini/immagine.jpg}
\caption{Esempio di figura centrata e con width 0.25}
\end{figure}



\begin{figure}[h!] 
\centering
\includegraphics[width=0.5\linewidth]{immagine.jpg}
\caption{Esempio di figura con width 0.5}
\end{figure}

\begin{itemize}
  \item La keyword \verb|centering| centra l'immagine nel documento.
  \item La keyword \verb|width| definisce quanto grande è l'immagine.
\end{itemize}


Con il comando \verb|\newpage| è possibile definire e cominciare una nuova pagina del documento.
\newpage
% --- TABELLE ---
\section{Creare tabelle}

Le tabelle si creano con il comando \verb|tabular|.

Sintassi:
\begin{verbatim}
\begin{tabular}{formato}
contenuto
\end{tabular}
\end{verbatim}
Formato colonne:
\begin{itemize}
  \item \verb|l| sinistra \quad
  \item \verb|c| centro \quad
  \item \verb|r| destra \quad
  \item \verb||| linea verticale
\end{itemize}
Dentro la tabella:
\begin{itemize}
  \item \verb|&| separa colonne \quad
  \item \verb|\\| fine riga \quad
  \item \verb|\hline| linea orizzontale
\end{itemize}
Esempio:

\begin{table}[h!]
\centering
\begin{tabular}{l|r}
Oggetto & Quantità \\\hline
Widget & 42 \\
Gadget & 13
\end{tabular}
\caption{Esempio di tabella}
\end{table}

\verb|table| serve per centrare e aggiungere la didascalia; la struttura è in \verb|tabular|.


\newpage
\section{Formule matematiche}

\LaTeX{} permette di scrivere formule in modo chiaro e professionale.

\textbf{Inline}: si usa \verb|$ ... $|  
Esempio: $a^2 + b^2 = c^2$

\textbf{Display}: si usa \verb|\[ ... \]|  

\[
a^2 + b^2 = c^2
\]

\subsection*{Comandi principali}

\begin{itemize}
\item Apice: \verb|^| \quad $x^2 \quad x^{10}$
\item Pedice: \verb|_| \quad $a_1 \quad a_{ij}$
\item Frazione: \verb|\frac{a}{b}| \quad $\frac{a}{b}$
\item Radice: \verb|\sqrt{x}| \quad $\sqrt{x}$
\item Sommatoria: \verb|\sum_{i=1}^{n}| \quad $\sum_{i=1}^{n}$
\item Integrale: \verb|\int_0^1| \quad $\int_0^1 x^2 dx$
\end{itemize}

\subsection*{Lettere greche}

Si scrivono con \verb|\| + nome:

\[
\alpha \quad \beta \quad \gamma \quad \lambda \quad \pi
\] 

\subsection*{Equazioni allineate}

Con il pacchetto \verb|amsmath| si usa l’ambiente \verb|align|:

\begin{verbatim}
\begin{align}
a &= b + c \\
  &= d + e
\end{align}
\end{verbatim}


\end{document}