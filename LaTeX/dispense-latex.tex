\documentclass[a4paper,12pt]{article}

% Lingua e codifica
\usepackage[italian]{babel}
\usepackage[utf8]{inputenc}
\usepackage[T1]{fontenc}
\usepackage{lmodern}
\usepackage{graphicx}

% Impaginazione
\usepackage{geometry}
\geometry{margin=2.5cm}
\usepackage{hyperref}
\usepackage{booktabs}
\usepackage{array}
\usepackage{xcolor}

% Comandi utili
\newcommand{\code}[1]{\texttt{#1}}
\newcommand{\comando}[1]{\texttt{\textbackslash #1}}

\begin{document}

\begin{titlepage}
\centering

{\Large I.I.S. \textit{``Francesco Alberghetti''} Imola}\\[0.3cm]
{\normalsize Liceo Scientifico Scienze Applicate}\\[2.2cm]

\rule{0.85\textwidth}{0.8pt}\\[0.9cm]

{\huge\bfseries Introduzione a \LaTeX}\\[0.4cm]
{\Large Scrivere documenti con un linguaggio di markup}\\[0.6cm]

\rule{0.85\textwidth}{0.8pt}\\[2.5cm]

{\Large Materiale didattico di Informatica}\\[0.3cm]

\vfill

{\large Prof. Federico Mazzini}\\[0.3cm]
{\large Anno scolastico 2025 -- 2026}

\end{titlepage}

\newpage
\tableofcontents
\newpage

%%%%%%%%%%%%%%%%%%%%%%%%%%%%%%%%%%%%%%%%%%%%%%%%%%%%%%%%%%%%%%
\section*{Obiettivo}
%%%%%%%%%%%%%%%%%%%%%%%%%%%%%%%%%%%%%%%%%%%%%%%%%%%%%%%%%%%%%%

Comprendere che \LaTeX{} non è solo uno strumento di scrittura,  
ma un primo contatto con il concetto di linguaggio formale,  
sintassi e compilazione.

\bigskip

%%%%%%%%%%%%%%%%%%%%%%%%%%%%%%%%%%%%%%%%%%%%%%%%%%%%%%%%%%%%%%
\section{Che cos'è \LaTeX}
%%%%%%%%%%%%%%%%%%%%%%%%%%%%%%%%%%%%%%%%%%%%%%%%%%%%%%%%%%%%%%

\LaTeX{} è un sistema per la produzione di documenti basato su un linguaggio testuale.  
Non si scrive ``trascinando'' elementi con il mouse, ma scrivendo \textbf{comandi}.

Un documento \LaTeX{} è un semplice file di testo:
\begin{itemize}
    \item contiene parole;
    \item contiene comandi;
    \item viene compilato per produrre un PDF.
\end{itemize}

Questo approccio è molto diverso da un editor WYSIWYG (come Word), ma è estremamente potente.

\subsection{Perché usare \LaTeX}

\begin{itemize}
    \item separa contenuto e forma;
    \item rende il documento coerente e strutturato;
    \item introduce alla logica dei linguaggi formali;
    \item abitua a scrivere seguendo regole sintattiche precise.
\end{itemize}

Scrivere in \LaTeX{} significa imparare a dialogare con un sistema che interpreta istruzioni.

\section{Un linguaggio, non un programma grafico}
%%%%%%%%%%%%%%%%%%%%%%%%%%%%%%%%%%%%%%%%%%%%%%%%%%%%%%%%%%%%%%

\LaTeX{} è un linguaggio di markup. Al contrario degli editor basati sul paradigma \textbf{WYSIWYG} (``What You See Is What You Get''), come ad esempio Microsoft Word, Google Docs, o LibreOffice Writer, con \LaTeX{} non si modifica direttamente l'aspetto grafico del testo. \\
Negli editor WYSIWYG:
\begin{itemize}
    \item si sceglie un font;
    \item si imposta una dimensione (es. 54 pt);
    \item si applica il grassetto;
    \item si interviene sulla forma visiva del testo.
\end{itemize}
In \LaTeX{}, invece:
\begin{itemize}
    \item si specifica che un testo è un \textbf{titolo};
    \item si indica che un blocco è una \textbf{sezione};
    \item si descrive la \textbf{struttura logica} del documento.
\end{itemize}
In Latex non si dice \emph{come deve apparire} il testo,  
ma \emph{che cosa rappresenta} quel testo.

Il file latex è quindi un puro file testuale, con estensione .tex, che contiene sia il contenuto che le istruzioni necessarie per ottenere l'output finale. Esiste poi un programma, chiamato compilatore che prende il testo Latex e decide come deve apparire graficamente, creando poi un PDF.

Questo modello è simile alla programmazione:
\begin{itemize}
    \item esiste una sintassi da rispettare;
    \item esistono regole formali;
    \item esiste un interprete (compilatore) che produce un risultato.
\end{itemize}

Scrivere in \LaTeX{} significa quindi descrivere il significato del testo,  
non la sua estetica immediata.

%%%%%%%%%%%%%%%%%%%%%%%%%%%%%%%%%%%%%%%%%%%%%%%%%%%%%%%%%%%%%%
\section{Struttura di un documento}
%%%%%%%%%%%%%%%%%%%%%%%%%%%%%%%%%%%%%%%%%%%%%%%%%%%%%%%%%%%%%%

Ogni documento \LaTeX{} ha una struttura precisa.

\subsection{Struttura minima}

\begin{verbatim}
\documentclass{article}

\begin{document}

Ciao mondo!

\end{document}
\end{verbatim}
Elementi fondamentali:
\begin{itemize}
    \item \comando{documentclass} definisce il tipo di documento;
    \item \comando{begin\{document\}} inizia il contenuto;
    \item \comando{end\{document\}} lo termina.
\end{itemize}

%%%%%%%%%%%%%%%%%%%%%%%%%%%%%%%%%%%%%%%%%%%%%%%%%%%%%%%%%%%%%%
\section{Comandi e sintassi}
%%%%%%%%%%%%%%%%%%%%%%%%%%%%%%%%%%%%%%%%%%%%%%%%%%%%%%%%%%%%%%

Un comando \LaTeX{}:
\begin{itemize}
    \item inizia con il simbolo \code{\textbackslash};
    \item può avere argomenti tra parentesi graffe \code{\{\}};
    \item può avere opzioni tra parentesi quadre \code{[]}.
\end{itemize}

Esempi:

\begin{verbatim}
\textbf{Testo in grassetto}
\section{Titolo}
\documentclass[12pt]{article}
\end{verbatim}
\textbf{La sintassi deve essere rispettata con precisione.  
Un errore di parentesi produce errore di compilazione.}





%%%%%%%%%%%%%%%%%%%%%%%%%%%%%%%%%%%%%%%%%%%%%%%%%%%%%%%%%%%%%%
\subsection{Strutturare il testo}

La struttura di un documento in \LaTeX{} è gerarchica.
Le sezioni non servono a modificare la dimensione del testo, 
ma a definire il ruolo logico delle parti del documento.

In alcune classi di documento, come \verb|book| o \verb|report|, 
è disponibile anche il comando \verb|\chapter|, che rappresenta 
un livello gerarchico superiore rispetto a \verb|\section|.

\bigskip
\paragraph{Comandi}

\begin{verbatim}
\chapter{Capitolo}
\section{Titolo}
\subsection{Sottotitolo}
\subsubsection{Paragrafo}
\end{verbatim}

\bigskip

\paragraph{Risultati}
\section*{Titolo}
\subsection*{Sottotitolo}
\subsubsection*{Paragrafo}

\bigskip

La gerarchia è quindi:

\begin{itemize}
    \item \verb|\chapter| livello più alto (se disponibile);
    \item \verb|\section| livello principale;
    \item \verb|\subsection| livello annidato;
    \item \verb|\subsubsection| livello ancora più specifico.
\end{itemize}

I comandi senza asterisco producono numerazione automatica e sono presenti all'interno dell'indice.
La versione con asterisco (es. \verb|\section*|) rimuove la numerazione.


\newpage
%%%%%%%%%%%%%%%%%%%%%%%%%%%%%%%%%%%%%%%%%%%%%%%%%%%%%%%%%%%%%%
\subsection{Formattazione del testo}
%%%%%%%%%%%%%%%%%%%%%%%%%%%%%%%%%%%%%%%%%%%%%%%%%%%%%%%%%%%%%%

In \LaTeX{} la formattazione non si applica con il mouse:
qualsiasi modifica del testo deve essere esplicitata tramite un comando.

Si vuole colorare una porzione di testo? Esiste un comando. Si vuole porre in grassetto? Esiste un comando. Si vuole sottolineare o enfatizzare? Anche questo si può fare.

\bigskip

\paragraph{Comandi principali}

\begin{verbatim}
\textbf{testo}        % grassetto
\textit{testo}        % corsivo
\texttt{testo}        % monospazio
\underline{testo}     % sottolineato
\emph{testo}          % enfatizzato (di solito corsivo)
\textsc{testo}        % maiuscoletto
\textsf{testo}        % font sans-serif
\textrm{testo}        % font roman
\textnormal{testo}    % font normale
\textcolor{red}{testo} % testo colorato (richiede xcolor)
\end{verbatim}

\bigskip

\paragraph{Risultato}\mbox{}\\

Questo è \textbf{grassetto}.  

Questo è \textit{corsivo}.  

Questo è \texttt{monospazio}.  

Questo è \underline{sottolineato}.  

Questo è \emph{enfatizzato}.  

Questo è \textsc{maiuscoletto}.  

Questo è \textsf{sans-serif}.  

Questo è \textcolor{red}{rosso}.  


\bigskip

\paragraph{Si possono anche combinari comandi}

\begin{verbatim}
\textbf{\textit{grassetto e corsivo}}
\textcolor{blue}{\underline{blu e sottolineato}}
\end{verbatim}
\textbf{\textit{grassetto e corsivo}}\\
\textcolor{blue}{\underline{blu e sottolineato}}


\newpage
\subsection{Elenchi numerati e puntati}

In \LaTeX{} gli elenchi si creano utilizzando ambienti dedicati.
La sintassi deve essere esplicitata nel codice.

\paragraph{Elenchi numerati}

\begin{verbatim}
\begin{enumerate}
    \item Primo elemento
    \item Secondo elemento
\end{enumerate}
\end{verbatim}

\paragraph{Risultato}

\begin{enumerate}
    \item Primo elemento
    \item Secondo elemento
\end{enumerate}

\bigskip

\paragraph{Elenchi puntati}

\begin{verbatim}
\begin{itemize}
    \item Punto 1
    \item Punto 2
\end{itemize}
\end{verbatim}

\paragraph{Risultato}

\begin{itemize}
    \item Punto 1
    \item Punto 2
\end{itemize}

\bigskip

\subsubsection{Elenchi annidati}

Gli elenchi possono essere inseriti uno dentro l'altro.
Ogni livello ha il proprio ambiente.

\paragraph{Sintassi}

\begin{verbatim}
\begin{itemize}
    \item Primo livello
    \item Ancora primo livello
    \begin{itemize}
        \item Secondo livello
        \begin{itemize}
            \item Terzo livello
        \end{itemize}
    \end{itemize}
    \item Di nuovo primo livello
\end{itemize}
\end{verbatim}

\paragraph{Risultato}

\begin{itemize}
    \item Primo livello
    \item Ancora primo livello
    \begin{itemize}
        \item Secondo livello
        \begin{itemize}
            \item Terzo livello
        \end{itemize}
    \end{itemize}
    \item Di nuovo primo livello
\end{itemize}

\newpage
\section{Inserire figure}

Per inserire immagini nel documento si utilizza l'ambiente \verb|figure|
insieme al comando \verb|\includegraphics|.

\paragraph{Sintassi base}

\begin{verbatim}
\begin{figure}[h!]
    \includegraphics{immagini/immagine.jpg}
    \caption{Didascalia della figura}
\end{figure}
\end{verbatim}

\paragraph{Esempio minimo}

\begin{figure}[h!]
    \includegraphics[width=0.25\linewidth]{immagini/immagine.jpg}
    \caption{Esempio di figura}
\end{figure}

\bigskip

\paragraph{Figura centrata}

\begin{verbatim}
\begin{figure}[h!]
    \centering
    \includegraphics[width=0.25\linewidth]{immagini/immagine.jpg}
    \caption{Figura centrata}
\end{figure}
\end{verbatim}

\paragraph{Risultato}

\begin{figure}[h!]
    \centering
    \includegraphics[width=0.25\linewidth]{immagini/immagine.jpg}
    \caption{Figura centrata}
\end{figure}

\bigskip

\paragraph{Dimensione dell'immagine}

\begin{verbatim}
\includegraphics[width=0.5\linewidth]{immagini/immagine.jpg}
\end{verbatim}

\begin{figure}[h!]
    \centering
    \includegraphics[width=0.5\linewidth]{immagini/immagine.jpg}
    \caption{Figura con width 0.5}
\end{figure}

\begin{itemize}
    \item \verb|\centering| centra l'immagine;
    \item \verb|width| controlla la dimensione dell'immagine;
    \item \verb|\caption| aggiunge una didascalia numerata.
\end{itemize}

Con il comando \verb|\newpage| è possibile forzare l'inizio di una nuova pagina.


\newpage
%%%%%%%%%%%%%%%%%%%%%%%%%%%%%%%%%%%%%%%%%%%%%%%%%%%%%%%%%%%%%%
\section{Creare tabelle}

In \LaTeX{}, le tabelle si creano principalmente con l'ambiente \verb|tabular|. 
L'ambiente \verb|table| serve per aggiungere didascalie, centrare la tabella e riferimenti automatici.

\subsection*{Teoria base}

\begin{itemize}
  \item Le colonne hanno un formato: \verb|l| (sinistra), \verb|c| (centro), \verb|r| (destra)
  \item Le linee verticali si ottengono con \verb+|+
  \item Il separatore di colonne è \verb|&|
  \item La fine di una riga si indica con \verb|\\|
  \item Linee orizzontali: \verb|\hline|
\end{itemize}

---

\subsection*{Sintassi generale}

\begin{verbatim}
\begin{tabular}{formato_colonne}
cella1 & cella2 & cella3 \\
cella4 & cella5 & cella6 \\
\end{tabular}
\end{verbatim}

Esempio minimo:

\begin{table}[h!]
\centering
\begin{tabular}{l|r}
Oggetto & Quantità \\\hline
Widget & 42 \\
Gadget & 13
\end{tabular}
\caption{Esempio di tabella}
\end{table}

---

\subsection*{Comandi principali e risultato}

\begin{itemize}
  \item \verb|l, c, r| \quad allineamento colonne: sinistra, centro, destra
  \item \verb+|+ \quad linea verticale tra colonne
  \item \verb|&| \quad separa le colonne
  \item \verb|\\| \quad fine riga
  \item \verb|\hline| \quad linea orizzontale
\end{itemize}

---

\subsection*{Esempi pratici}

\paragraph{Tabella semplice senza linee verticali}

\begin{verbatim}
\begin{tabular}{l r}
Nome & Età \\
Alice & 20 \\
Bob & 22
\end{tabular}
\end{verbatim}

\paragraph{Risultato}

\begin{tabular}{l r}
Nome & Età \\
Alice & 20 \\
Bob & 22
\end{tabular}
\bigskip


\paragraph{Tabella con linee verticali e orizzontali:}

\begin{verbatim}
\begin{tabular}{|l|c|r|}
\hline
Prodotto & Quantità & Prezzo \\
\hline
Widget & 42 & 1.50 \\
Gadget & 13 & 2.30 \\
\hline
\end{tabular}
\end{verbatim}

\paragraph{Risultato}

\begin{tabular}{|l|c|r|}
\hline
Prodotto & Quantità & Prezzo \\
\hline
Widget & 42 & 1.50 \\
Gadget & 13 & 2.30 \\
\hline
\end{tabular}
\bigskip


\newpage
%%%%%%%%%%%%%%%%%%%%%%%%%%%%%%%%%%%%%%%%%%%%%%%%%%%%%%
\section{Formule matematiche}

\LaTeX{} permette di scrivere formule matematiche in modo chiaro e professionale. 
Esistono due modalità principali: 

\begin{itemize}
  \item \textbf{Inline}: la formula è inserita nella stessa riga del testo. Si racchiude tra \$...\$.  
  Esempio: $a^2 + b^2 = c^2$
  
  \item \textbf{Display}: la formula occupa una riga propria e viene centrata. Si usa \verb|\[ ... \]| o \verb|equation|.  
  Esempio:
  \[
  a^2 + b^2 = c^2
  \]
\end{itemize}

\subsection*{Teoria base dei comandi matematici}

Ogni formula è costruita usando simboli speciali. Alcuni concetti chiave:

\begin{itemize}
  \item \textbf{Apice}: indica potenze, si scrive con \verb|^|
  \item \textbf{Pedice}: indica indici o sottomultipli, si scrive con \verb|_|
  \item \textbf{Frazione}: divide numeratore e denominatore, \verb|\frac{}{}|
  \item \textbf{Radice}: si usa \verb|\sqrt{}|
  \item \textbf{Sommatoria}: \verb|\sum| con pedici e apici
  \item \textbf{Integrale}: \verb|\int| con pedici e apici
\end{itemize}


\subsection*{Comandi principali e risultato}

\begin{itemize}
  \item Apice: \verb|x^2, x^{10}| \quad $\Rightarrow x^2, x^{10}$
  \item Pedice: \verb|a_1, a_{ij}| \quad $\Rightarrow a_1, a_{ij}$
  \item Frazione: \verb|\frac{a}{b}| \quad $\Rightarrow \frac{a}{b}$
  \item Radice: \verb|\sqrt{x}| \quad $\Rightarrow \sqrt{x}$
  \item Sommatoria: \verb|\sum_{i=1}^{n}| \quad $\Rightarrow \sum_{i=1}^{n} i$
  \item Integrale: \verb|\int_0^1 x^2 dx| \quad $\Rightarrow \int_0^1 x^2 dx$
  \item Radice n-esima: \verb|\sqrt[3]{x}| \quad $\Rightarrow \sqrt[3]{x}$
  \item Simboli di disuguaglianza: \verb|<, >, \le, \ge| \quad $\Rightarrow x < y, x \ge y$
\end{itemize}


\subsection*{Lettere greche}

Si scrivono usando \verb|\| + nome.  

\begin{verbatim}
  \alpha, \beta, \gamma, \lambda, \pi, \theta, \Delta
\end{verbatim}
Risultato:

\[
\alpha, \beta, \gamma, \lambda, \pi, \theta, \Delta
\]


\subsection*{Altri simboli utili}

\begin{itemize}
  \item $\infty$ infinito
  \item $\approx$ circa uguale
  \item $\forall, \exists$ quantificatori universale e esistenziale
  \item $\rightarrow, \leftarrow, \leftrightarrow$ frecce
\end{itemize}


\newpage

%%%%%%%%%%%%%%%%%%%%%%%%%%%%%%%%%%%%%%%%%%%%%%%%%%%%%%%%%%%%%%
\section{Documenti modulari in \LaTeX}
%%%%%%%%%%%%%%%%%%%%%%%%%%%%%%%%%%%%%%%%%%%%%%%%%%%%%%%%%%%%%%

\LaTeX{} permette di creare documenti modulari, separando la struttura (header) dal contenuto vero e proprio.  
Questo approccio facilita modifiche e riutilizzo del materiale.

\subsection*{Idea base}

\begin{itemize}
    \item Creare un file \textbf{header.tex} che contiene tutti i comandi e il contenuto delle prime pagine:
    \begin{itemize}
        \item il tipo di documento (\code{\textbackslash documentclass});
        \item pacchetti utilizzati;
        \item definizioni di comandi personalizzati;
        \item prima pagina e indice.
    \end{itemize}
    \item Creare un file \textbf{main.tex} (o altro nome) che contiene:
    \begin{itemize}
        \item il corpo del documento (\code{\textbackslash begin\{document\}} ... \code{\textbackslash end\{document\}});
        \item capitoli, sezioni, testo, figure, tabelle, formule.
    \end{itemize}
\end{itemize}

\subsection*{Come collegare i file}

Nel file \textbf{contenuto.tex}, inserire all'inizio del documento:

\begin{verbatim}
\input{header.tex}
\begin{document}
...
\end{document}
\end{verbatim}

In questo modo, qualsiasi modifica fatta nel file \textbf{header.tex} si riflette automaticamente in tutti i documenti che lo includono.

\subsection*{Vantaggi}

\begin{itemize}
    \item Contenuto principale separato dall'intestazione: facile da scrivere e leggere;
    \item Possibilità di cambiare tipo di documento e prima pagina in un solo posto senza toccare il contenuto del documento;
    \item Facilità di mantenere più versioni dello stesso documento cambiando solo l'header;
    \item Maggiore chiarezza nella struttura del documento.
\end{itemize}

\subsection*{Suggerimento pratico}

Si possono avere più header, ad esempio:

\begin{itemize}
    \item \textbf{header\_saggio.tex} per esperimenti di laboratorio;
    \item \textbf{header\_lezione.tex} per appunti o dispense;
    \item \textbf{header\_relazione.tex} per relazioni finali.
\end{itemize}

Basta cambiare il file incluso tramite \code{\textbackslash input} e il contenuto si adatta automaticamente.

\newpage
%%%%%%%%%%%%%%%%%%%%%%%%%%%%%%%%%%%%%%%%%%%%%%%%%%%%%%%%%%%%%%
% Pagina titolo esercizi
\thispagestyle{empty}
\begin{center}
\vspace*{8cm} % centra verticalmente
{\Huge \textbf{Esercizi}}
\end{center}
\newpage

%%%%%%%%%%%%%%%%%%%%%%%%%%%%%%%%%%%%%%%%%%%%%%%%%%%%%%%%%%%%%%
\section*{Esercizi su formule matematiche}

\begin{enumerate}
    \item Scrivere la legge di Ohm:
    \[
    V = R \cdot I
    \]
    dove $V$ è la tensione, $R$ la resistenza e $I$ la corrente.

    \item Scrivere la formula dell’energia cinetica:
    \[
    E_c = \frac{1}{2} m v^2
    \]
    con $m$ massa e $v$ velocità dell’oggetto.

    \item Scrivere la formula della densità:
    \[
    \rho = \frac{m}{V}
    \]
    dove $m$ è la massa e $V$ il volume.

    \item Scrivere la legge di Hooke per le molle:
    \[
    F = k \cdot \Delta x
    \]
    dove $F$ è la forza, $k$ la costante elastica e $\Delta x$ l’allungamento.

    \item Scrivere l’equazione chimica bilanciata per la combustione del metano:
    \[
    CH_4 + 2 O_2 \rightarrow CO_2 + 2 H_2O
    \]

    \item Scrivere la legge della gravitazione universale di Newton:
    \[
    F = G \frac{m_1 m_2}{r^2}
    \]
    con $G$ costante di gravitazione, $m_1$, $m_2$ masse dei corpi e $r$ distanza tra i centri.
\end{enumerate}

\newpage
%%%%%%%%%%%%%%%%%%%%%%%%%%%%%%%%%%%%%%%%%%%%%%%%%%%%%%%%%%%%%%
\section*{Algoritmi e complessità}

Un algoritmo è una sequenza finita di istruzioni che descrive come risolvere
un problema in modo preciso e non ambiguo.  
Gli algoritmi sono alla base di tutti i programmi informatici.

\bigskip

\begin{itemize}
    \item \textbf{Definizione di algoritmo}  
    Un algoritmo deve essere:
    \begin{itemize}
        \item finito, cioè terminare dopo un numero limitato di passi;
        \item deterministico, ogni istruzione è univoca;
        \item generale, valido per tutti gli input ammessi.
    \end{itemize}

    \item \textbf{Esempio di problema}  
    Calcolare la somma dei primi $n$ numeri naturali:
    \[
    S(n) = 1 + 2 + 3 + \dots + n
    \]

    \item \textbf{Soluzione algoritmica}  
    La somma può essere calcolata con la formula:
    \[
    S(n) = \frac{n(n+1)}{2}
    \]

    \item \textbf{Costo computazionale}  
    Oltre alla correttezza, un algoritmo si valuta per il numero di operazioni
    necessarie a produrre il risultato.

    \item \textbf{Schema riassuntivo}
    \begin{itemize}
        \item gli algoritmi risolvono problemi;
        \item la correttezza è fondamentale;
        \item l’efficienza è misurabile;
        \item la complessità permette il confronto.
    \end{itemize}
\end{itemize}

\bigskip

\begin{figure}[h!]
    \centering
    \includegraphics[width=0.45\linewidth]{immagini/algorithm.jpeg}
    \caption{Immagine ad effetto per rappresentare un algoritmo}
\end{figure}


\newpage
%%%%%%%%%%%%%%%%%%%%%%%%%%%%%%%%%%%%%%%%%%%%%%%%%%%%%%%%%%%%%%
\section*{Alberi}

Un albero è una struttura dati gerarchica utilizzata per rappresentare
relazioni di tipo padre--figlio.  
Un albero non è lineare, ma organizzato su più livelli. 

\begin{figure}[h!]
    \centering
    \includegraphics[width=0.3\linewidth]{immagini/albero.png}
    \caption{Esempio di albero}
\end{figure}

\begin{itemize}
    \item \textbf{Definizione}  
    Un albero è composto da:
    \begin{itemize}
        \item un nodo radice (root);
        \item archi che collegano ogni nodo padre con i propri nodi figli;
        \item nodi foglia (leaf), che non hanno figli.
    \end{itemize}

    \item \textbf{Tipologie}  
    Un albero può essere:
    \begin{itemize}
        \item \textbf{ordinato}: la relazione tra padre e figlio è regolata da una qualche logica.
        \item \textbf{non ordinato}: non esistono regole nella relazione padre figlio.
    \end{itemize}

    Una speciale categoria di strutture ad albero sono gli alberi binari, in cui ogni nodo può avere al massimo due figli, un figlio di sinistra e uno di destra.
    
    In un albero binario di altezza $h$, il numero massimo di nodi è:
    \[
    N_{\max} = 2^{h+1} - 1
    \]

    \item \textbf{Applicazioni}  
    Gli alberi hanno molte applicazioni nel campo informatico:
    \begin{itemize}
        \item pagine web;
        \item organizzazione di file e cartelle nel sistema operativo;
        \item editor di testo come Word e LibreOffice;
        \item database e dati.
    \end{itemize}
\end{itemize}

\bigskip

\end{document}